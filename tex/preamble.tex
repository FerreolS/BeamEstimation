%
%      Copyright (C) Ferréol Soulez, 2020 
%      ferreol.soulez@univ-lyon1.fr
%
%     This program is free software: you can redistribute it and/or modify
%     it under the terms of the GNU General Public License as published by
%     the Free Software Foundation, either version 3 of the License, or
%     (at your option) any later version.
%
%     This program is distributed in the hope that it will be useful,
%     but WITHOUT ANY WARRANTY; without even the implied warranty of
%     MERCHANTABILITY or FITNESS FOR A PARTICULAR PURPOSE.  See the
%     GNU General Public License for more details.
%
%     You should have received a copy of the GNU General Public License
%     along with this program.  If not, see <http://www.gnu.org/licenses/>.
%
%% Ferréol Soulez, 2020

\usepackage{amsmath,amsfonts,amssymb,textcomp,mathtools}
\usepackage{mathrsfs}
\usepackage{xspace} 


\makeatletter \newcommand{\MathFuncName}[1]{{\operator@font #1}}
\newcommand{\MathFunc}[1]{\mathop{\operator@font #1}\nolimits}
\newcommand{\MathFuncWithLimits}[1]{\mathop{\operator@font #1}\limits}
\makeatother

\newcommand{\asterisk}{*}
\newcommand{\mathd}{\mathrm{d}} % roman 'd' for integrand/differential
\newcommand{\mathe}{\mathrm{e}} % roman 'e' for exp(1)
\newcommand{\mathi}{\mathrm{i}} % roman 'i' for sqrt(-1)

\newcommand{\BF}[1]{\mathbf{#1}} % math bold
\newcommand{\RM}[1]{\mathrm{#1}} % math roman
\newcommand{\SF}[1]{\mathsf{#1}} % math sans-serif
\newcommand{\OP}[1]{{\MathFuncName{#1}}} % upright (*) for function names
\newcommand{\Tag}[1]{\mathsf{#1}} % math font for labels and tags
\newcommand{\BS}[1]{{\boldsymbol{#1}}} % bold symbol (*)
\newcommand{\V}[1]{\boldsymbol{#1}} % vector
\newcommand{\M}[1]{\mathbf{#1}} % matrix
\newcommand{\Inner}[2]{\left \langle #1,#2 \right\rangle} % Inner product
% \newcommand{\TransposeLetter}{\mathrm{T}}
% \newcommand{\TransposeLetter}{\mathsf{t}}
% \newcommand{\TransposeLetter}{{\intercal}}
% \newcommand{\TransposeLetter}{\top}
\newcommand{\AdjointLetter}{\dagger}
\newcommand{\T}{^{\AdjointLetter}} % suffix for transpose
\newcommand{\TransposeLetter}{\top}
\newcommand{\Tt}{^{\TransposeLetter}} % suffix for transpose
\newcommand{\mT}{^{-\TransposeLetter}} % suffix for inverse transpose
\newcommand{\I}{\mathi} % roman 'i' for sqrt(-1)
\newcommand{\E}{\mathe} % roman 'e' for exp(1)
\newcommand{\D}[1]{{\mathd #1}} % integrand/differential
\newcommand{\ScaleAs}{\mathcal{O}} % O(n)
\renewcommand{\Re}{\MathFunc{Re}} % real part
\newcommand{\sign}{\MathFunc{sign}} % sign
\newcommand{\Arg}{\MathFunc{Arg}} % argument
%\newcommand{\mod}{\MathFunc{mod}} % modulo
\renewcommand{\Im}{\MathFunc{Im}} % imaginary part
\newcommand{\arc}{\MathFunc{arc}} % lenght of arc
\newcommand{\rect}{\MathFunc{rect}} % real part
\newcommand{\rnd}{\MathFunc{rnd}} % random
\newcommand{\sgn}{\MathFunc{sgn}} % sign function
\newcommand{\sinc}{\MathFunc{sinc}} % cardinal sine
\newcommand{\asin}{\MathFunc{asin}} % inverse sine
\newcommand{\acos}{\MathFunc{acos}} % inverse cosine
\newcommand{\Var}{\MathFunc{Var}} % variance
\newcommand{\Cov}{\MathFunc{Cov}} % covariance
\DeclarePairedDelimiterXPP{\Expect}[1]{\mathrm{E}}(){}{#1}
\newcommand*{\delimsize}{}
\newcommand*{\given}{\,\vert\,} % compact version
\newcommand*{\Given}{\:\delimsize\vert\:} % autoscale to surrounding version

\newcommand{\Card}{\MathFunc{Card}} % cardinal, number of elements
\newcommand{\diag}{\MathFunc{diag}} % diagonal
\newcommand{\eig}{\MathFunc{eig}} % eigenvalues
\newcommand{\tr}{\MathFunc{tr}} % trace
\newcommand{\rank}{\MathFunc{rank}} % rank
\newcommand{\pdf}{\MathFunc{pdf}} % probability density function
\newcommand{\para}{{/\!/}} % parallel
\newcommand{\prox}{\operatorname{prox}} % proximity operator
\newcommand{\conj}[1]{\operatorname{conj}\Paren{#1}} % proximity operator
\newcommand{\proxy}[1]{\widetilde{#1}} % approximation
% (*) enclosed in braces so that x_\BS{u} or x_\RM{name} works
\newcommand{\comp}{\mathop{\circ}}



% Partial derivative:
\newcommand{\PDer}[2]{\frac{\partial #1}{\partial #2}}
\newcommand{\At}[2]{\left.#1\right\vert_{#2}}

\newcommand{\Paren}[1]{\left(#1\right)}
\newcommand{\bigParen}[1]{\bigl(#1\bigr)}
\newcommand{\BigParen}[1]{\Bigl(#1\Bigr)}

\newcommand{\Brace}[1]{\left\{#1\right\}}
\newcommand{\bigBrace}[1]{\bigl\{#1\bigr\}}
\newcommand{\BigBrace}[1]{\Bigl\{#1\Bigr\}}

\newcommand{\SqBrack}[1]{\left[#1\right]}
\newcommand{\bigSqBrack}[1]{\bigl[#1\bigr]}
\newcommand{\BigSqBrack}[1]{\Bigl[#1\Bigr]}

% Norm, e.g. ||x|| (fixed and scalable):
\newcommand{\norm}[1]{\Vert #1\Vert} 
\newcommand{\Norm}[1]{\left\Vert #1\right\Vert}
\newcommand{\bigNorm}[1]{\bigl\Vert #1\bigr\Vert}
\newcommand{\BigNorm}[1]{\Bigl\Vert #1\Bigr\Vert}

% Absolute value, e.g. |x| (fixed and scalable):
\newcommand{\abs}[1]{\vert #1\vert} 
\newcommand{\Abs}[1]{\left\vert #1\right\vert} \newcommand{\bigAbs}[1]{\bigl\vert #1\bigr\vert}
\newcommand{\BigAbs}[1]{\Bigl\vert #1\Bigr\vert}

% Average value, e.g. <x> (fixed and scalable):
\newcommand{\avg}[1]{\langle #1\rangle}
\newcommand{\Avg}[1]{\left\langle #1\right\rangle}
\newcommand{\bigAvg}[1]{\bigl\langle #1\bigr\rangle}
\newcommand{\BigAvg}[1]{\Bigl\langle #1\Bigr\rangle}

% Complex quantity:
\newcommand{\complex}[1]{\underline{#1}}

% For optimization problems, "arg min" and "arg max":
\newcommand{\argmin}{\MathFuncWithLimits{arg\,min}}
\newcommand{\argmax}{\MathFuncWithLimits{arg\,max}}
\newcommand{\minimize}{\MathFuncWithLimits{minimize}}
\newcommand{\find}{\MathFuncWithLimits{find}}

% Marker for end of proof/algorithm:
\newcommand{\EndProof}{\ensuremath{\blacksquare}}
% \newcommand{\EndProof}{\ensuremath{Box{}}} Quadratic term:
\newcommand{\QuadTerm}[2]{#2\T\!\!\cdot #1\cdot #2}

% Fourier transforms:
\newcommand{\FT}[1]{\widehat{#1}} % Fourier transform
\newcommand{\TwoIPi}{2\,\I\,\pi} % 2*i*pi
\newcommand{\F}[1]{\mathcal{F}\Paren{#1}}
\newcommand{\IF}[1]{\mathcal{F}^{-1}\Paren{#1}}
\newcommand{\FTarrow}{\xrightarrow{\mathrm{FT}}}
\newcommand{\DFTarrow}{\xrightarrow{\mathrm{DFT}}}

% Sets:
\newcommand{\Set}[1]{\mathbb{#1}} \newcommand{\Reals}{\mathbb{R}}
\newcommand{\Integers}{\mathbb{Z}} \newcommand{\Natural}{\mathbb{N}}
\newcommand{\Complexes}{\mathbb{C}}
\newcommand{\Rationals}{\mathbb{Q}}

% By definition: \newcommand\bydef{\equiv}
% \newcommand\bydef{\triangleq}
\newcommand{\bydef}{\stackrel{\text{\tiny def}}{=}}

\newcommand\BesselJ[1]{\mathop{\mathrm{J}}\nolimits_{#1}}


\renewcommand{\Re}{\MathFunc{Re}}        % real part
\renewcommand{\Im}{\MathFunc{Im}}        % imaginary part


\newcommand{\Eq}[1]{Eq.~(\ref{#1})}
\newcommand{\Fig}[1]{Fig.~\ref{#1}}


% Common abbreviation:
\newcommand{\cf}{\emph{cf.}\xspace}
\newcommand{\eg}{\emph{e.g.}\xspace}
\newcommand{\ie}{\emph{i.e.}\xspace} \newcommand{\etal}{\emph{et
    al.}\xspace} \newcommand{\etc}{\emph{etc.}\xspace}
\newcommand{\wrt}{with respect to\xspace}
